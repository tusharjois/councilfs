\documentclass{article}


\usepackage{arxiv}

\usepackage[utf8]{inputenc} % allow utf-8 input
\usepackage[T1]{fontenc}    % use 8-bit T1 fonts
\usepackage{hyperref}       % hyperlinks
\usepackage{url}            % simple URL typesetting
\usepackage{booktabs}       % professional-quality tables
\usepackage{amsfonts}       % blackboard math symbols
\usepackage{nicefrac}       % compact symbols for 1/2, etc.
\usepackage{microtype}      % microtypography
\usepackage{lipsum}

\title{CouncilFS: An Incentivized, Resilient File Distribution System}


\author{
  Gabrielle Beck \\
  Department of Computer Science\\
  Johns Hopkins University \\
 Baltimore, MD \\
  \texttt{becgabri@jhu.edu} \\
  %% examples of more authors
   \And
 Tushar Jois \\
  Department of Computer Science\\
  Johns Hopkins University\\
  Baltimore, MD \\
  \texttt{jois@cs.jhu.edu} \\
  %% \AND
  %% Coauthor \\
  %% Affiliation \\
  %% Address \\
  %% \texttt{email} \\
  %% \And
  %% Coauthor \\
  %% Affiliation \\
  %% Address \\
  %% \texttt{email} \\
  %% \And
  %% Coauthor \\
  %% Affiliation \\
  %% Address \\
  %% \texttt{email} \\
}

\begin{document}
\maketitle

\begin{abstract}
  We present \textsc{CouncilFS},
\end{abstract}


% keywords can be removed
%\keywords{First keyword \and Second keyword \and More}


\section{Introduction}


The main purpose of this work is to try and solve the inherent wastefulness in
popular cryptocurrencies by proposing a proof of work that can be used to
perform a useful function, in this case, storing files for those who are willing
to pay for such storage. Other blockchain researchers before us have also
identified this as a problematic issue and have tried to rectify it through
creating alternatives to Bitcoin’s hash-based puzzle. These solutions include
schemes that do not rely on ludicrous amounts of computation such as Proofs of
Stake (where who gets to mine is based off of players that have more stock in
the current network) and Proofs of Space (where there is a lot of computation
done up front for initialization of a new miner, but the actual mining itself is
cheap). However, schemes that do not rely on heavy computation tend to suffer
from “nothing at stake” problems because there is no incentive for miners to
work on just the longest chain when it is so inexpensive to try and extend other
forks.

Therefore, the other alternative to these style of proofs are proofs of
work that serve some utilitarian function (i.e. distributed file storage).
Proofs of Retrievabiliy (PoRs) have been proposed as alternatives to a hash
based proof of work by authors such as the creators of Permacoin, however, there
scheme has some limitations in that they assume that the file has to be publicly
available and that it can be handed out to new miners by a trusted dealer. This
limits its usefulness. Because they focus on storing a large publicly known
file, they also do not acknowledge any potential that this scheme brings for
loose censorship resistance. By distributing the trusted dealer among more
mutually distrusting parties, we can accommodate the storage of encrypted files
and provide a myriad of parties that a client can access should any few be
blocked or taken down. Overall, we believe our scheme leverages economic
incentives and blockchain technologies to create a currency that can more or
less be used to drive distributed storage in a way that penalizes those that
attempt to cheat clients out of money without of their files and allows for
discrete sharing of potentially covert information.

\paragraph{Contribution.} We present \textsc{CouncilFS}, a file distribution
system with censorship-resistant properties. These properties are incentivized
with a blockchain-based system that provides payment for distribution services
rendered. Our contributions are the following:
\begin{itemize}
\item A new file distribution system that uses a two-level node hierarchy to
  bootstrap trust and a blockchain to provide incentives.
\item A protocol that combines proofs-of-retrieveability, with the
  notion of payment channels to provide incentivized 
\item The implementation the proofs-of-retrieveability, client, and alderman code of
  \textsc{CouncilFS} in the Go programming language. 
\end{itemize}


\section{Related Work}
\subsection{Alternative Proofs of Work}
Our work exists at the cross-section of many different active fields of research. Before this paper, there have been many proposals for Proof-of-Work schemes that serve some useful function or purpose like PrimeCoin. In a previous paper, there has also been a proposal to utilize a POR as a proof of work for the purpose of storing archival information that would be known by all miners so that validation of the files could be assured $\mathcal{D}$. The authors also assume that this file would not be updated and introduce a new stateful signature scheme that should be relatively efficient while also ensuring that a miner cannot "outsource" their computation. This differs from our scheme as we allow for the existence of other files carried by clients that are given directly to trusted nodes, called \emph{Alderman}, and are not verifiable by every single node on the network.\\

Other alternative proofs of work that are similar to PORs include Proofs of Space, Proofs of Data Possession and Proofs of Erasure. Proofs of Space, for example, guarantee that a miner is holding some space it could be doing computation open and then answer challenges to ensure this is in fact the case. The way this scheme is commonly implemented is to use a form of pebbling and pebble the spaces and prove something about the graph to a verifier. One popular Proof of Space is spacemint which is unique in that it implements two different blockchains and utilizes a punishment mechanism to disincentivize miners from deviating from the protocol in ways that are unlikely to happen if Proof of Work algorithms (for example, nothing at stake problems and block/challenge grinding). In our future work, we propose a similar kind of punishment mechanism. However, we use this tool to demote misbehaving \emph{Alderman} and to provide recompense for grief-stricken clients. We do not propose two different blockchains/ledgers for history either. 

\subsection{Censorship Resistant File Storage}
There has also been work done previously in the creation of file storage resistant to the takedown of particular governments and agencies. Publius, for example, was proposed as a method of resistance in which n servers host the encrypted content but in order to get the key a client must obtain k-out-of-n secret shares that are also controlled by these servers. While there is plausible deniability on the part of the servers, as they are hosting encrypted material, there is no mechanism or recompense in place for servers electing to delete their shares and no way to communicate this information to all of the other servers readily on the network. What this paper attempts to do is fundamentally different as we not only want to provide a mechanism by which servers can help dissidents/clients but we want to provide them an incentive to do so as well. Namely, in the form of currency. 

\section{Construction}

\subsection{Overview}

There are three types of nodes in our network: \emph{members}, \emph{aldermen},
and \emph{clients}. Members form the backbone of the network; they are the nodes
that primarily interface with the blockchain. They verify transactions between
users of the underlying currency of \textsc{CouncilFS}. Aldermen are members of the
network who have been elected to take on more responsbility in the network.
Their primary responsbility is to maintain instances of the files that clients
upload. Clients pay for this privlege, and engage in a protocol to ensure that
the alderman remains honest. Aldermen who are not honest are demoted back to the
member pool. This general outline of our construction is discussed in more
detail in the subsections below.

\subsection{Members}

At the core of the \textsc{CouncilFS} network are members. The underlying token of
\textsc{CouncilFS}, known as a \emph{shire}, is exchanged between users as transactions.
Members organize user transactions into blocks, verify correctness, and perform
a proof-of-work computation. The computation performed is then compared to a
difficulty value by all other nodes, and if it is less than this value, the rest
of the network accepts a member's block. This process, known as Nakamoto
Consensus, is analgous to the process performed by miners in Bitcoin \cite{btc}.

\subsubsection{Proofs-of-Retrievability}

Unlike Bitcoin miners, however, \textsc{CouncilFS} members perform a
proof-of-retrieveability (PoR), in which a prover proves that they are able to
retrieve a file on demand, without transmitting the entire file to a the
verifier. This PoR is performed by every member on a common dataset $\mathcal{D}$; because
this dataset is common, each other member in the network acts a verifier. The
PoR is computationally intensive, and is still fundamentally a proof-of-work
scheme; it acts as a replacement for the SHA-256 hash function used in Bitcoin
and related cryptocurrencies.

Our instantiation of PoR is similar to the one in Permacoin \cite{perma}, in
that it acts as a positive use of computational resources. There are some key
differences, however. In \textsc{CouncilFS}, the aldermen replace the dealer as the
arbiter of the common dataset. The aldermen decide which members get which
slices of $\mathcal{D}$, rather than having a trusted setup phase. Additionally, because
the aldermen handle $\mathcal{D}$, it can be updated continuously. Both of these points
are discussed further later in the paper.

\subsection{Aldermen}

Aldermen constitute the second tier of nodes in \textsc{CouncilFS}. These nodes are
trusted by the others in the network to provide file distribution and
logistics services for all of \textsc{CouncilFS}. File storage capabilities are provided
to \emph{clients}, who pay the aldermen for this distribution service. Other
nodes in the network can download files from the aldermen as long as this
payment is made. This extra network responsibility is bootstrapped by the escrow
of shire tokens and by a selection process. 

\subsubsection{Selection}

\subsubsection{Punishment}

An alderman is expected to behave according to the norms of the protocol. When
an alderman starts misbehaving, it is in the best interest of the rest of the
alderman if they catch the misbehavior (assuming the majority of aldermen are
honest). A deeper analysis of the incentives surrounding alderman behavior is in
the Discussion section below.

Aldermen determine misbehavior by challenging each other on the operations of
the network. If an alderman can prove to a challenger that it is meeting the
requirements of the protocol, it can remain in the network; otherwise, it is
punished by being sent back to the member pool. Implicit to this challenge
process is the determination of liveness, as a dead alderman will not respond to
a challenge fast enough.

\subsection{File Distribution}

\subsubsection{Common Data Store $\mathcal{D}$}

\subsubsection{Client File Channels}

\subsubsection{File Retrieval}

\section{Evaluation}

\subsection{Implementation}

\subsection{Microbenchmarks}

\section{Discussion}

\paragraph{Incentive Structure.}

\paragraph{Censorship Resistance.}

\paragraph{Trusting the Aldermen.}

\paragraph{Updating $\mathcal{D}$.}

\section{Conclusion}

\section{Examples of citations, figures, tables, references}
\label{sec:others}
 \cite{permacoin-repurposing-bitcoin-work-for-data-preservation}

The documentation for \verb+natbib+ may be found at
\begin{center}
  \url{http://mirrors.ctan.org/macros/latex/contrib/natbib/natnotes.pdf}
\end{center}
Of note is the command \verb+\citet+, which produces citations
appropriate for use in inline text.  For example,
\begin{verbatim}
   \citet{hasselmo} investigated\dots
\end{verbatim}
produces
\begin{quote}
  Hasselmo, et al.\ (1995) investigated\dots
\end{quote}

\begin{center}
  \url{https://www.ctan.org/pkg/booktabs}
\end{center}


\subsection{Figures}
\lipsum[10] 
See Figure \ref{fig:fig1}. Here is how you add footnotes. \footnote{Sample of the first footnote.}
\lipsum[11] 

\begin{figure}
  \centering
  \fbox{\rule[-.5cm]{4cm}{4cm} \rule[-.5cm]{4cm}{0cm}}
  \caption{Sample figure caption.}
  \label{fig:fig1}
\end{figure}

\subsection{Tables}
\lipsum[12]
See awesome Table~\ref{tab:table}.

\begin{table}
 \caption{Sample table title}
  \centering
  \begin{tabular}{lll}
    \toprule
    \multicolumn{2}{c}{Part}                   \\
    \cmidrule(r){1-2}
    Name     & Description     & Size ($\mu$m) \\
    \midrule
    Dendrite & Input terminal  & $\sim$100     \\
    Axon     & Output terminal & $\sim$10      \\
    Soma     & Cell body       & up to $10^6$  \\
    \bottomrule
  \end{tabular}
  \label{tab:table}
\end{table}

\subsection{Lists}
\begin{itemize}
\item Lorem ipsum dolor sit amet
\item consectetur adipiscing elit. 
\item Aliquam dignissim blandit est, in dictum tortor gravida eget. In ac rutrum magna.
\end{itemize}


\bibliographystyle{unsrt}  
\bibliography{references}  %%% Remove comment to use the external .bib file (using bibtex).
%%% and comment out the ``thebibliography'' section.


%%% Comment out this section when you \bibliography{references} is enabled.
%\begin{thebibliography}{1}

%\bibitem{kour2014real}
%George Kour and Raid Saabne.
%\newblock Real-time segmentation of on-line handwritten arabic script.
%\newblock In {\em Frontiers in Handwriting Recognition (ICFHR), 2014 14th
%  International Conference on}, pages 417--422. IEEE, 2014.

%\bibitem{kour2014fast}
%George Kour and Raid Saabne.
%\newblock Fast classification of handwritten on-line arabic characters.
%\newblock In {\em Soft Computing and Pattern Recognition (SoCPaR), 2014 6th
%  International Conference of}, pages 312--318. IEEE, 2014.

%\bibitem{hadash2018estimate}
%Guy Hadash, Einat Kermany, Boaz Carmeli, Ofer Lavi, George Kour, and Alon
%  Jacovi.
%\newblock Estimate and replace: A novel approach to integrating deep neural
% networks with existing applications.
%\newblock {\em arXiv preprint arXiv:1804.09028}, 2018.

%\end{thebibliography}


\end{document}
